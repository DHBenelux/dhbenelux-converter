The \emph{Graph Poem} project, started at University of Ottawa in 2014
and continued meanwhile both there and at UCLouvain, has developed
computational poetry classifiers that are deployed in representing
poetry corpora as network graphs which are, in their turn, analyzed for
graph theory-reliant features that will reflect back on the corpora and
the poems thereof. This paper is part of a forthcoming cluster of
publications that takes our focus beyond the strict affiliation of
computational analysis, natural language processing (NLP), and
graph-theory applications and into a wider digital humanities (DH)
context. This complementing direction explores the poetry and poetics of
DH and a possible ``plus-poetics'' as manifest in programs and
performances like ours or David Jhave Johnston's \emph{Big Data Poetry}
(\cite{tanasescu2021poetryand}), the tightly-knit intercorrelation
between digital writing, control, and ``monstrous iconicity'' in digital
space and media (\cite{tanasescu\&tanasescu2021text}), and the present
writing aiming to foreground the social and community relevance and
impact in digital space-based performances such as the Margento
\#GraphPoem EPoetry events presented at Digital Humanities Summer
Institute (DHSI) in 2019 and 2020.\footnote{See
  \href{https://bit.ly/2ASWDYl}{{https://bit.ly/2ASWDYl}}; the recording
  of the \#GraphPoem @ DHSI 2020 performance can be watched here:
  \href{https://bit.ly/2Nmk5j1}{{https://bit.ly/2Nmk5j1}}; for the
  automated \#GraphPoem tweets at DHSI 2019 and DHSI 2020 search for the
  @GraphPoem bot's profile page on Twitter.}

One of the concerns in our previous NLP publications was to make a point
in favor of poetry's relevance beyond the genre per se, in the wider
framework of NLP and DH. For instance, the metaphor classifiers we
developed trained on poetry and non-poetry data turned out to be better
than the ones trained by other authors on just the latter or by us on
just the former (\cite{kesarwani2017metaphor}), which also helped with a
deep-learning approach to metaphor (\cite{tanasescu2018metaphor}), while
important NLP instruments such as word embeddings trained on poetry
corpora turned out to be better for any other text analysis purposes
than other `household name' ones (\cite{tanasescu2018metaphor}).

In the present paper, we want to extend that argument regarding poetry's
relevance `beyond poetry' to the realm of the social and the political.
In that respect, we find Jonas Andersson Schwarz's revisiting of the
concept of \emph{Umwelt} (in Uexkull's philosophy,
cf.~\cite{schwarz2018umwelt}) quite useful for our argument as it
foregrounds nuanced notions of milieu and ecology that can correlate
communities or social assemblies to various other kinds of ensembles.
Digital space-based poetic corpora, for instance, are such ensembles,
alongside the social ones, and can be better understood and worked on in
contexts transgressing commonly accepted borders between the organic and
the technical, as well as between the human and the machinic. In
translating this notion to our world of artifacts, Schwarz draws on
Johansson to highlight the fact that artefacts become ecological
entities only as long as they are ``attended to and used,'' thus
becoming ``as much an agent (co-agent) in the social ecology as is the
organic human being'' (\cite{schwarz2018umwelt}, p.~66). We are
particularly enticed by this fundamentally operational (or rather, as
argued below, \emph{performative)} and socially connecting nature of
artifacts, especially in a framework whereby, in the footsteps of Alfred
Gell, ``the mediated environment prompts human self-understanding to
take on \emph{mental categorizations} that are isomorphic to this
environment'' (\cite{schwarz2018umwelt} 61, author's emphasis).

We take this (``human'') self-understanding to be the poem's
(self-)reading and (self-)performance as informed by its corpora and
corporeal environment. Moreover, we see the corpus-based performance of
the poem and the networked textualities thereof as potentially
isomorphic to the societal connectivity and the radical approaches to
community they are shaped by and/or feed back into. Our very concept of,
and actual term, the \emph{graph poem} (singular) refers to the
multitude of poems in a corpus algorithmically analyzed and expanded as
a network (i.e., mathematical `graph') that together amount to, or
rather asymptotically tend towards, a (locally) global encompassing
poem. At the same time, every single node in such a network---every
single poem indeed in the corpus---is informed and performed by that
specific surrounding milieu of various (globally) local
communities/subnetworks, or Umwelt(s).

We will involve three key agents in our notion of performative networked
sociality of poetry in digital culture: humans, poems (as performative
inscriptions in digital space)\footnote{We are using the phrase
  ``digital space'' in Stephen Kennedy's acceptance of the term, the
  (ubiquitous/all-engulfing) space of chaos media that ontologically
  goes beyond the real-virtual binary and that is informed by a
  non-representational paradigm and by the ``sonic economy'' best
  describing digital culture (\cite{kennedy2015chaos}).}, and machines
(as computationally implemented algorithms and artificial intelligence).
While we do not believe that these agents are one and the same, we
maintain that they deeply and intimately overlap in respects remarkably
relevant to the (post)digital. And that overlapping is in our view made
possible by the ``isomorphism,'' in the quotation above, between
``mental categorizations'' and the mediated environment. Still, while
those categorizations are for Schwarz---via Gell---instrumental in
``human'' ``self-understanding,'' for us they are rather the very
vehicle for shaping isomorphic networked milieus of
humans/poems/machines. And in doing so, they shape the networked
individuals across these interconnected milieus as well. It is in fact
off of such isomorphisms that we base our notion of the potentially
radical societal and community-oriented relevance of poetry in digital
space and culture. Our umbrella term for such relevance is
\emph{data-commoning webformance,} which we will detail in a bit.

But before doing that, let us look more closely into the symmetrical
bi-directional process we just alluded to, whereby environments and
individuals are shaped and shape each other simultaneously. We concur in
this perspective with authors in the field of constructivist theories of
cognition in which the crucial hypothesis is that ``living systems are
not primarily defined through the discrete qualities of their
components, but through relations'' (\cite{schwarz2018umwelt} 66). Our
model is consistent with such theories in that we see poems---in digital
media---as never in isolation, but always inscribed in the medium as
performative and contextual. Their constitutive features are always
informed by the (computational processing of the) other poems in the
relevant corpus/ora, just as they are performed by the other writing
operations involved in the inscription---and, again, computational
processing---of those corpora in digital space and media. In terms of
digital writing---as computational inscription and processing---the poem
performs its environment while being itself performed by the latter, or,
in the language of the above-mentioned theories, ``the object comes to
appear as if it generates an Umwelt'' (\cite{schwarz2018umwelt}).

We would like to extend the scope of this model towards correlationist
philosophy and explore the latter's previous translation to the subject
of poetry---mainly in Brian Kim Stefans's \emph{Word Toys. Poetry and
Technics} (\cite{stefans2017word})---and thus consider it as a possible
basis for discussing poetry's potentially radical societal and
community-oriented relevance. Stefans's poetics and references,
remarkably rich and far too complex to be fully addressed in a
discussion like the present one, will nevertheless provide more
opportunities to explore than just the correlationist one. Yet for the
latter already, the way in which Stefans draws on philosophers such as
Quentin Meillassoux and writers like Vilém Flusser in, for instance,
highlighting the notion that ``only the correlation of the mind and
object is what matters---neither can be understood without the other''
(\cite{stefans2017word}, p.~1-2) will prove truly pertinent to the point
we want to make.

We are (re)reading that latter statement from the dual angle of the
isomorphisms above and the computational correlations informing our
graph theory-based model. Our contention is that ``mind-object''
correlations, that is, connections made possible by means of (mutual)
reading/processing/performance, refer to any (two or more) nodes in our
networks. Poems inscribed and computationally processed in digital media
are therefore understandable---they indeed can only exist
actually---only through the correlations between one another. Every
single poem is the mind while all the others are that mind's objects.

Further on, correlate sets of ``mind categorizations'' will ensure the
propagation of the intra- or inter-corpus correlations into the social.
Such processual and performative model involves manifold feedback
circuits: the societal impact will reflect back on the graph poem's
dataset(s), (re)configurations, and dynamics, which in turn will (help
to) revisit, reshuffle, and at times reformulate the mathematics and
algorithmics behind it, which in turn will impact the technology,
platform, and web-based venue choices and/or constraints, and so on and
so forth.

We are far from pioneering in trying to conceptualize poems as actors in
a wider onto(techno)logical universe (or, rather, multiverse) shaping
their own milieus. Although the actual societal values of that is
different in his vision than ours, Brian Kim Stefans says something
consistent on at least a couple of levels: ``I'd like to imagine poems
as autonomous entities that, like machines and living organisms, enact
their own interactions with their milieus, perhaps each with its own
`will to power' and desire to reproduce, obtain sustenance, and evolve''
(\cite{stefans2017word}, p.~2).While we are not attracted to the concept
of the poem's autonomy and its being a ``non-textual and even
non-cultural object'' (\cite{stefans2017word}, idem), we are indeed of
the same mind in terms of a deep similarity between poems, on the one
hand, and machines and ``living organisms,'' on the other, especially in
their interactive shaping of (and, naturally, being shaped by) their
milieus.

Stefans draws in fact on major 20th century philosophers, perhaps most
predominantly on Simondon and the latter's philosophy of technics, to
spectacularly apply such approaches in poetry. He notably borrows
Simondon's breakdown of technics into elements, individuals, and
ensembles, where elements have no actual functional autonomy (e,g.
wheels), individuals represent coherent and autonomous assemblages of
elements (e.g.~locomotives and cars), while ensembles link several
objects (elements, individuals, etc.) into chains of production (cf.
\cite{stefans2017word}, p.~61 et infra). Elements are characteristic of
the whole pre-industrial era, the rise of industry coincides with the
rise of individuals, and ensembles represent the technology of the
``information age,'' the laboratory and, we would say most
significantly, the computer network (idem).

While he is obviously not the first to explore the technical and,
moreover, technological nature of poetry---major modernists such as,
most popularly William Carlos Williams, but also Ezra Pound, have
started a tradition that culminated with the rise of digital
poetry---Stefans, while an outstanding practitioner of the latter, is
the first to pursue this notion into developing a large-scale
extensively theoretical and intensively illustrative poetics that refers
to poetry in general; inclusive, that is, of both traditional
`page-based' as well as (post)digital (sub)genres. As already alluded, a
work of such complexity and scope cannot possibly be properly discussed
within the space and time of this writing, and that is why we will limit
ourselves to highlighting some of the main concepts and particularly the
ones relevant to our own topic and approach.

``I would like to suggest that poems, particularly the lyric (or short
poem), `succeed' to the level that they approach something like the
technicity of a technical individual'' (\cite{stefans2017word}, p.~69)
states Stefans opting for the individual on Simondon's historical and
typological scale. Poems are in his view therefore non-machinic
coming-to-terms with the ``presence of technical essences,'' and, again
in the French philosopher's terms, the continuation of life ``by means
other than life'' (\cite{stefans2017word}, idem). One may find striking
this preference of the poet for the individual, but in fact, Stefans
revisits the philosopher's classification and sees in the poem an
ensemble ``elevated {\ldots} to a technical individual'' that, just
like a machine, has parts performing functions of interaction
with\ldots{} \emph{a milieu,} in ways similar to Pound's own
``condensare'' (cf.~\cite{stefans2017word}, idem).

It is on this preference for the technical individual that incorporates
features of the ensemble, particularly in interacting with the milieu,
that Stefans bases his appropriation of Simondon's distinctions as
adaptable to poetry. His own resulting innovative vision establishes,
like above, tie-ins with certain foundational tenets of modern poetry
and poetics. And as we will see below, this particular way of adapting
such philosophical concepts to the subject will also allow Stefans to
further tap into Simondon's notion of individuation and discover more
valences relevant to, and useful for, poetry there as well. But we need
first to outline his own vision on the milieu and its relevance to the
social in as much as poetry is concerned.

Stefans argues therefore that, since, as Simondon elucidates, technical
individuals have the highest degree of technicity, they actually present
the evolution of machines towards what is ``nearly organic'' (63,
author's emphasis). Consequently, their constitutive elements, the
``{i}nfra-individual technical objects,'' (\cite{stefans2017word},
idem) have no associated milieu, as they simply work as parts of the
individual ``like the heart or liver in the body.'' While the
constitutive elements have no milieu, the individual, the genuine
invention, will characteristically require one, since its relationship
to the latter are not mediated by any other technical entities.
Translated for poetry, this means, very much in line with Stefans's
poetics, that poems are self-contained technical objects with components
(``elements'') that interact only between themselves and that, as a
whole, do not necessarily have a society around them. Neither the
society of other poems (Stefans speaks little of corpora, and poems
relate to each other mainly through an inherent generative diagram, when
and if the latter is the same\footnote{Although this may of course
  remind one of New Criticism, Stefans's is a rather radical political
  and literary vision (inspired for example by Alain Badiou in ideology
  and the pataphysicians in literature) and more redolent of the
  Californian anarchist tradition rather than other more (or over)
  orthodox lineages.}), nor human society.

The latter aspect is in and of itself at the crux of Stefans's poetics.
As he frontally announces from his introduction and already briefly
mentioned here, to him poems are ``non-textual and even non-cultural
objects'' (\cite{stefans2017word}, p.~2), a perspective that is totally
consistent with his reading of Simondon's ideas on the development of
technics as having its own unique track of evolution. ``apart from (and
certainly not dependent upon) social or economical developments, one
that can literally outpace culture's ability to absorb these changes
into art, philosophy, or behaviors\ldots{}'' (\cite{stefans2017word},
p.~60) Does this mean that he has no interest in the social dimension of
poetry? That is not the case at all, yet his is---in this respect as
well---a rather one-of-a-kind approach.

There are two features of the poem that have or can acquire a societal
impact for Stefans. The first one refers to the fact that poems can work
as \emph{pharmakon} (Stiegler's concept based on Simondon's
individuation) and are \emph{evental} (Badiou's term for revolution as
singularity creating new possibility from the void, and thus paving the
way for the impossible). It is the former that is relevant to our
discussion, so we will briefly outline the concept and then look
critically into its possible interconnections with the computational
poetry performance in our focus.

To Simondon (as revisited by Stefans) individuation is a person's growth
into a singularity by tapping into their pre-individual grounds that can
work as a, if not \emph{the}, source of creativity
(cf.~\cite{stefans2017word}, idem). Interestingly enough, the social
valence of that emerges as a form of therapy, social therapy consisting
in the rapprochement between humans and technical objects, a notion that
has been developed by Stiegler into his own well-known concept of
\emph{pharmakon}. It is the latter that Stefans appropriates and applies
in his discussion of poetry, thus foregrounding the poem as the
(rediscovered/recuperated) toy that can liberate us from the stinted,
narrowly pre-defined, and alienating social relationships we are
entrapped in. In doing so, pharmakon can help us reconnect with, or even
regain, infinity as existential amplitude and unfettered sociality.
While revisiting Anne Waldman's verse and using such reading keys, for
instance, Stefans excavates from her verse poetic strategies to
``establish the long-circuit of fidelity to the milieu'' (p.~97) and
thus illustrate Stiegler's pharmakon and the very definition of
``infinite thought'' at the same time. (98)

It is important to note that Stefans's mediation of Simondon's thought
is itself an assertion of the poet's own poetics and consequential to
the way in which he articulates the (non) sociality of poetry. His
adaptation and development of the concept of pharmakon for poetry
criticism is based on a definition of Simondon's individuation that
comes with an impactful reduction. It is limited to ``persons'' (and
their growth and becoming creative), which inevitably reduces (the)
sociality (of poems) to human society, while in fact the philosopher
conceived of individuation as applying ``to molecules, human beings,
technical objects, and collective societies alike''
(\cite{schwarz2018umwelt}, p.~63).

Poems therefore, as technical objects (or rather, as Stefans strongly
argues, technical individuals) can also know individuation, and they do
so in, and by means of, interacting with their milieus. And, as stated
earlier on, a model informed by environment-sensitive (mental)
categorizations can ensure isomorphisms (or at least consistent and
chartable correlations) between various kinds of milieus, in this
particular case, networks of poems and online communities involved in
data-intensive networked computational poetry performance. The Margento
\#GraphPoem EPoetry event at DHSI 2019 was announced as a performance
and consisted of a JupyterHub Python notebook available on the
University Victoria server only for the duration of the event (Figure
{1} (\#fig:Tanasescu\_et\_al\_Figure\_1)\{reference-type=``ref''
reference=``fig:Tanasescu\_et\_al\_Figure\_1''\})---authored by Chris
Tanasescu---for collective live data collection, code running, and
output visualization, and a bot (@GraphPoem)---programmed by Prasadith
Buddhitha---that tweeted content outputted by the script on JupyterHub
combined with text visualization and YouTube videos (Figure {2}
(\#fig:Tanasescu\_et\_al\_Figure\_2)\{reference-type=``ref''
reference=``fig:Tanasescu\_et\_al\_Figure\_2''\}). The latter featured
cross-artform work by Margento, the performance poetry band whose name
has been transferred over time to an international collective of
writers, artists, coders, and translators doing collaborative
writing/art/performance, as well as to the team working on the DH
project \#GraphPoem. The participants in DHSI and the concurrent ADHO
SIG DH Pedagogy Conference could access the JupyterHub script and run it
on a corpus of poems assembled by us yet also available for them too to
enlarge (with basically any text they found relevant, interesting,
appealing, and/or simply random).

\begin{figure}
\hypertarget{fig:Tanasescu_et_al_Figure_1}{%
\centering
\includegraphics{article-39-tanasescu/Images/Tanasescu_et_al_Figure_1.png}
\caption{\#GraphPoem @ DHSI 2019 JupyterHub
Notebook.{}}\label{fig:Tanasescu_et_al_Figure_1}
}
\end{figure}

\begin{figure}
\hypertarget{fig:Tanasescu_et_al_Figure_2}{%
\centering
\includegraphics{article-39-tanasescu/Images/Tanasescu_et_al_Figure_2.png}
\caption{The @GraphPoem bot tweeting at DHSI
2019.{}}\label{fig:Tanasescu_et_al_Figure_2}
}
\end{figure}

Although the event had been announced as a performance, the participants
and other people in the audience were for quite some time confused and
waiting for the `gig' to start. Still, as the bot was intermittently
tweeting and using a couple of relevant hashtags, more and more people
started getting push notifications and at times nudging each other,
``look, it actually already started, it's on Twitter, etc.'' As the
tweets came with visualizations and/or videos, soon participants,
guests, and spectators starting following or watching the same or, most
often, different things as simultaneously as part of the same event.
These various mediations and temporalities and the interactions they
spawned amounted to an enactment of the model on various levels and to a
(number of) multilayer performance(s).

The term for such interactive community-oriented digital space-based
performance as advanced in a previous publication is ``commoning'' (see
\cite{tuanuasescucommunity}). The term draws, on the one hand, on
contemporary (post-Occupy) radical (`strike') art and/as performance,
and, on the other, on recent radical thought (Hardt and Negri, for
instance, and their description of the occupying multitude as a
performance, cf.~\cite{tuanuasescucommunity} p.~12). Moreover, it also
gestures towards non-essentialist approaches to community (Agamben's
``coming community,'' \cite{tuanuasescucommunity}. pp.~20-2), and thus
refers to (re)shaping/sharing/founding/finding community in/as communal
enactment and/or collaborative performance (of the `commons'). In
articulating the latter aspect, we rely on recent advances in
performance (and memory) studies and practice, particularly the work of
Mechtild Widrich on ``performative monuments'' and reperformance (cf.
\cite{tanasescu\&tanasescu2021text}).

In our particular case---in as much as the event under discussion is
concerned but also for \#GraphPoem as the overarching project---the
commoning is enacted in three major way. First, by means of shared and
collectively expanded data (in the form of txt file corpora as well as
NLP and network generation and analysis algorithms in Python code),
second, by live interactive coding script running at a `commons' on
JupyterHub, and third, the algorithmic `communal occupation' of Twitter.
We see all of these components and activities as fundamentally
performative, particularly since digital space-based and all of them
informed by digital writing (in its turn, essentially performative, see
\cite{tanasescu\&tanasescu2021text}).

Performance still, as already suggested, spills into quite a number of
other aspects, media, and interrelations in potentially relevant
societal ways. The manifold interactions of the
performers/audience---between themselves and/or/by means of the various
computational components, media, temporalities, and
activities---represents to us a good opportunity to explore `in action'
the earlier on discussed isomorphisms between various entities such as
humans, poems, and machines engaged in (milieu-driven/shaping)
individuation in digital space and media.

We are particularly preoccupied with the individuation---and the
inextricably intercorrelated Umwelt(s)---of poems. From the experience
of events such as the one mentioned and the ongoing work with, and on,
poems and text in digital media, a reality---with its attending
poetics---emerges sensibly different from the one depicted by Stefans. A
poem does not exist in digital media otherwise but as contextual
inscription (enacted/performed always in relation to other texts and/or
digital writing operations) and variable instantiating readings (read as
instantiation{s} of various contextually/operationally relevant
features). In our particular case, including a poem in the DHSI event
corpus already deployed a number of other digital writing operations
related to the generation/instantiation and the format and location of
the directory. On JupyterHub, on the other hand, the script kept
`reading' the new files \emph{as part of the expanding corpus},
therefore as related to other (existing or potential) items in that
directory. Furthermore, as the participants run the script, the latter
will process the texts and mine them for features establishing the
correlations needed for representing the corpus as a network. In Lori
Emerson's terms (\cite{emerson2014reading}), the machine thus performs a
reading-writing that maps every poem for the relevant features and also
charts the whole corpus as informed by the interrelated quantifications
of those features for all items (poems/texts) contained.

A poem accordingly evolves towards its own individuation by being
integrated into the milieu which it shapes in its own turn by means of
its own process, or performance, of reading-writing/writing-reading. As
seen above, according to Simondon, this evolution can only take place by
tapping into the `raw' pre-individual `matter' within and without the
respective entity. In our case, as the poem is inscribed in digital
media and digital space by and for computational processing algorithms,
that said `matter' consists of quite a number of layers, stages,
parameters, and operations. Choices or default settings can be highly
consequential for instance in terms of the kind of character encoding
used (that makes `common' alphabetical and possibly other types of
characters readable to, and writable by, the machine) and other
mechanisms of embedding a sequence of characters as text in digital
media. Other settings or operations making for instance the file (and
its path) accessible to the subsequent computational processing, or
related to the algorithm(s) opening and reading the file (what is it
they filter out or not in terms of characters, spaces, line-breaks,
etc.) will impact significantly the processing and its output. At least
equally consequential will be the choices related to the NLP sets and
settings, the feature extraction and automated analysis (what kind of
tokenization, what does the stop word set consist of, what kind of
vectorization the script calls with what values for the parameters
involved, etc). All of the above constitute poem's pre-individuation
`matter' or materials in digital media and digital space.\footnote{In
  this particular case, the juxtaposition of the latter two
  environments---media and space (of the digital)---refers to a number
  of aspects among which the fact that since in digital culture, or the
  ``culture of connectivity'' (\cite{van2013culture}), nothing and
  nobody ever actually goes `off the grid' (cf.~for instance
  Kirschenbaum 2016), files written and/or processed on
  personal/self-contained machines/directories are always infiltrated
  (when not totally generated and contained) by elements (apps, word
  processors, programs, coding libraries, etc.) based in digital space,
  and therefore actually totally immersed in the latter given its
  operability and ontology. Moreover, both the `commons' (or `base of
  operations')---JupyterHub---and the `performance
  venue'---Twitter---used for that particular event are literally
  Web-based.}

It is such media and computation-related materials and factors that are
instrumental in inscribing and performing the poem in digital space and
in its contribution to the generation and (re)shaping of its milieu
and/as itself. The resulting milieu of poems and texts represent just a
few levels of that multilayer Umwelt in which all of those computational
and medial elements are active (and re/de-)formed in their turn. An
important aspect here, and as already mentioned above, sensibly
different from Stefans's vision (via Simondon), is the role of
computational features in processing the poems and/into their poetry
corpus milieu. In equating the poem with a technological individual,
Stefans also translates the particulars of the latter's interaction with
the environment, as reflected on its constitutive elements, onto the
poem as well. As already briefly explained, this positions the said
elements very much like the ``heart or liver in the body''
(\cite{stefans2017word}, p.~63) and therefore, and perhaps most
importantly, as having ``no associated milieu'' and ``not
interact{ing} with an environment outside of the machine''
(\cite{stefans2017word}, idem). The machine here is, of course, the
technological individual or, in Stefans's extrapolation, the poem.
Still, in computing data features for the poem's and the corpus'
processing and analysis it is exactly the constitutive (technological)
elements of the poem that enact its interaction with, and shaping of
(while being shaped by), the milieu. If in one layer of the multiplex
network, for instance, we represent correlations between the tf-idf
vectors of each and every poem, that will reflect the frequency of
certain terms in each of the poems as measured against their frequency
across the corpus. Then, if in another one, we represent the
correlations between the vectors representing rhyme and euphony scores
for every poem, that will reflect the intimate sonic anatomy of all
poems and the types and density of rhymes and other euphonious devices
across the corpus. It is not only that the heart and the liver of the
poem interact with the environment, it is actually through them alone
(and its other `internal organs') that the poem interacts with,
inhabits, shapes, and is shaped by, the environment.

There is, on the other hand, significant hybridization between the
poetic and the non-poetic in the traditional sense of the word---or
perhaps, apoetic---in the above outlined processes. In generating the
graph, a tf-idf layer may be significant but definitely applies to any
other genres and types of text as well. Also, as new files are added by
participants to the dataset, all sorts of text enrich the corpus, and
the poetic-feature classifiers run on them will just provide irregular
or erratic output. Other poetry-driven concerns will impact the NLP
framework and results.

For instance, we significantly tempered with the Python NLTK (natural
language toolkit)\footnote{{www.nltk.org}} stop word set as we wanted
for instance the first personal pronoun not to be flushed out in
tokenizing the texts. The first person singular is
traditionally---although at times rather stereotypically---seen as
regularly frequent in lyric poetry. Yet while certain poets can indeed
use it really frequently, others can consistently, sometimes even
blatantly, avoid it. Among the former there will be an iconic lyric poet
such as Sappho, to pick just a most famous example. But keeping the `I'
in will definitely help to identify the most representative of the
latter category as well. Emily Dickinson or Georg Trakl, for instance,
if in the corpus, will strikingly stand out. Dickinson's metaphysical
compression and symbolic scope goes beyond the individual in gnostic and
gnomic ways, while Trakl's expressionist visions are either too
alienating or too encompassing and ecstatic to allow or care about any
`I' submerged in them. We also chose not to consider the first person
plural a stop word. Its occurrence, particularly as weighed against the
singular, can suggest a nuanced and more or less self-equated take on
the community, as is the case with the poetries of Walt Whitman or Lyn
Hejinian.

We did not want adverbs of place either to be counted for stop words
especially since having a special interest in place poetry. But while
the concern above can have a more general genre-related relevance and
impact, this latter one was more of a task specific one. Besides being
the world-renowned institution, DHSI also means to the members of the
relevant communities, quite a number of personal and locative aspects
that very interestingly complement its explicit and perhaps prevailing
digital media and digital space focus. There are its charismatic and
intellectually outstanding organizers and there is also the indelibly
picturesque location, Victoria, B.C. Our option regarding such stop
words was therefore meant not only to ensure the importance of place as
event venue and genre-relevant stylistic feature at the same time, but
to also capture intra- and inter-textual characteristics or ambiguities
related to being digital space based and specifically geographically
located at the same time. Most importantly, perhaps, to unveil textual
corpus-imbedded features involved in distinguishing between the two
and/or perceiving them as inextricably interrelated.

From a more general perspective, such fine-tuning of the stop word set
(just as a host of other NLP-relevant choices and specific approaches)
profoundly inform the inscription, processing, and analysis of poems
within corpora. While this reflects a `philosophy' of poetry reading,
perhaps a poetics, on the part of the programmer, it does so in ways
that most intimately fuse the computational and the poetic: the poems
are the mode in which the machine reads and writes them, while the
machine is \emph{the} milieu of the poem. Such corollary stemming from
NLP, and more generally, computational programming instrumentality will
help us circle back to our main tenet regarding `the beyond' of poetry
as the poem's Umwelt and the humans-poems-machines ontological and
operational commonality it implicates.

There are in that respect two main aspects to note regarding the
inscriptive performance of poems (and texts networked into `graph
poems') in digital space and media. First, it is, in Simondon's (and
post-Uexkul) philosophical, as well as Stefans's literary theoretical,
terms here revisited from our own angle, precisely the technical
elements that prevent the poem from becoming or remaining a technical
individual and that \emph{are} yet instrumental in its individuation.
The technical---medial and computational---`apoetic' features and
procedures that shape it as a poem in its digital milieu. `Apoetic' here
has a twofold tenure, technological data-science features and approaches
that push or ignore the established/`traditional' boundaries of the
poetic genre, and also, technology that involves no \emph{poiesis}, no
putting forth of any coherent self-contained `individual' or `oeuvre'.
And, second in the above initiated enumeration, the technics making up
that specific digital milieu of the poem undergo in their turn a process
of individuation, as poetic commoning technology.

The sociality of poems and technologies is already there with its
hybridizing and performative nature even before explicitly involving
them in a `performance' event. And they are there with their live (as in
both living and at the actual time of occurrence/performance)---if not
human---component already as well, as, according to Stefans via
Simondon, ``continuation of life `by means other than life'\,''
(\cite{stefans2017word}, p.~69). Even before engaging in a
poetry-technology-driven commoning event, the poem is already
individuated/ing as machinic and human when impacting humans and the
interconnectivity between humans, just as the latter have already been
re/de-formed by, and into, technologies or poems. That is where we find
the social impact of poetry and/as technology at one of its possible
peaks, with individuation and milieu involving the processually
overlapping humans (as technological and poetic beings), poems (as
performative inscriptions in digital space), and machines (as
computationally implemented algorithms and artificial intelligence).

A previous initiative whereby we sought to demonstrate and enact the
societal relevance and impact of computational poetry (and which the
DHSI event actually built on) was the computationally assembled poetry
anthology \emph{``US'' Poets Foreign Poets} (\cite{margento2018us}).
While starting off by representing and analyzing an initial editorially
selected corpus of contemporary U.S. poetry as a network graph the
anthology advanced by algorithmically expanding it with poems that met
certain diction-related criteria and, consequently, held certain
peculiar positions in the gradually enlarged graph
(cf.~\cite{margento2019us}). Those newly added poems were included
strictly based on the above mentioned features ---having therefore a
diction that conferred them certain topological prominence in the
network---and irrespective of the author's region, thus opening the
selection to anybody whose poem(s) fit the unusual profile, and
translating the ``U.S'' in the title into ``us,'' poets elsewhere and
anywhere. This societal explicit implication and subversion of customary
editorial politics of exclusion was noticed by critics and practitioners
both on the social-literary level---Christopher Funkhouser, for
instance, posed the rhetorical question ``{w}ho among us ever
dreamed that we'd see an anthology where Alan Sondheim's work resides
near that of Charles Wright and Rita Dove, and in fact gets more page
space than they do?'' (\cite{funkhouser2019margento})---as well as the
communal-ontological one (made evident by the ``translation as process''
and generative graph informing the collection), as John Cayley noted
that the approach manages to transform ``U.S. into `US' because reading
and translating in this way is what makes us us''
(\cite{cayley2019ifiam}).

The DHSI 2019 event added to all of the above (and the book's
challenging of medial essentialism inhabiting the current hardcoded gaps
between print and digital) a `commoning' approach involving live the
community per se and resulting in a \emph{webformance.} The corpus
expanded by means of the corporeal and networked algorithmics straddled
an online commons and a social media website it automatically inundated
in ways that pushed the boundaries of corporate managed pre-established
frameworks for `user content'. In `traditional' performance poetry the
body is used as a medium of communication and/or status asserting prop,
while physical artefacts are extensions of the body modulating the
message and regulating (controlling even) the audience's reactions
and/or participation in the performance (see for instance
\cite{novak2011live}, pp.~151-169). The artefact in our case is the set
of algorithms selecting the outstanding poems/texts contributed by the
participants to the corpus (by means of network visualization and
analysis), and sampling them (with further visualization and/or
video-audio material) on Twitter. In doing so, the artefact contributes
to generating a societal milieu, by bringing about ``quasi-Umwelts,''
while to ``a conscious observer'' such ``object comes to appear as if it
generates an Umwelt'' (\cite{schwarz2018umwelt}, p.~66). Our conscious
observers---or at least a part of them---were the very participants in
the event and the milieu creation. The transition from the
``quasi-Umwelt'' above to what ``appears'' to be (i.e., is performed as)
a full-fledged Umwelt marks in our view the participants' own
transitioning from ``observers'' to being themselves performatively
individuated. Their individuation was enacted by their involvement in
the corpus expansion, the visualization and analysis of the resulting
evolving network as collective assembly, and at times by being
highlighted as remarkable co-assembler of the networked corpus whose
contribution got sampled on Twitter.

We are not as naively utopian as to say that an experiment employing the
artefacts above turn the traditional performance poetry paradigm on its
head and thus, instead of controlling the audience, it liberates and
includes them as full and unrestrained co-authors of the performance.
Yet a couple of potential upsides could definitely be advanced here.
While, for instance, in certain approaches to traditional performance it
was more often than not strictly the performer's body that the corporeal
dimension of the event resided in, and the artefacts were their
exclusive props and emblems of authority/authorship, in our particular
case, there is implicit but effective collective corporeal engagement
with the corpus and the algorithms that thus become everybody's
performative artefacts.

That kind of engagement begs a quick note on embodiment. According to N.
Katherine Hayles, embodiment is (unlike the body) an always enmeshed and
\emph{contextual enactment} (cf.~\cite{hayles2008we}, p.~196), and in
our case, these elements are obviously demonstrated in the literal
networked contextuality of the corpus and the algorithm, as well as in
the performative enactment of individuation and milieu as mutually
generative. Moreover though, within such an event the enactment is
framed as \emph{webformance,} that is, a performance that, in
(re/de)forming the Web, has actually everything to do with processual
corpus commoning. And it is therefore the process, the commoning, that
gets embodied, amounting to a corpus corporeal embodiment that ranges
from literal body evocations (as in sampling a graphic epigram from the
\emph{Shanzhai Lyric} corpus) or displays (as in pulling Margento videos
off of YouTube) to visualizing the network at a certain stage of
collective engagement.

This latter aspect alludes to another possible advantage of such
approaches. Namely, the attempted transition from the
control-the-audience to the
systemic-control-subverting-participatory-audience paradigm. The former
refers to performances in which the `onstage' performer(s) (tend or need
to) control the audience in conventionally medium-oblivious approaches
and settings; the latter to events highlighting the inherent
performative and at the same controlling nature of the medium and
involving (sections of) the audience in commoning practices potentially
subverting systemic control.\footnote{We are definitely not saying that
  the boundary between the two paradigms is the
  traditional-(post)digital one. It is actually imperiously necessary to
  note even if only in passing that `page-based' poets have framed the
  potential social impact of poetry in terms strikingly similar to those
  in our discussion. Mainstream contemporary American poet and Kenyon
  Review editor David Baker recently stated, for instance, that ``I do
  think poetry, the best of it, can help to shape a person's mind, his
  or her being-in-the-world, his receptiveness or her openness and
  rigor'' (\cite{baker2019curious}). Another awarded poet, Claudia
  Rankine talks about poetry's social function: ``It's not arguing a
  point. It's creating an environment.'' (\cite{baker2019curious})} The
cultural and political context for the latter is the typically the one
of cultures of connectivity and digital space, where the anonymous
system is inescapable and control is ubiquitous since embedded in the
very medium and the networks per se (cf.~for instance
\cite{franklin2012cloud}, \cite{tanasescu\&tanasescu2021text}).
Therefore, while we could not realistically speak of escaping control,
we can talk about exposing it and even working with control against
control, a point we have developed in another forthcoming publication
(\cite{tanasescu2021poetryand}). The strategy in our case involved going
back and forth between two online platforms and using algorithms to
automatically inundate one of them with the collective data assembled
and analyzed on the other. Yet we are not simply talking about sampling
in social media certain data stored elsewhere (although storing and
generating data at a different web-based location accessible only to the
participants was of crucial importance to the subversiveness of the
experiment), nor about just creating a bot that will tweet samples off a
given text/corpus (although the deployment of algorithms that output
content and then tweet that content is also essential for the message
and the performance experience). We are talking about \emph{data
commoning,} algorithmic community building through shared (collection
of, work on, and enlargement of) data, in/as \emph{webformance,}
performance that deploys networks (or user managed webs) to expose the
Web's inherent control, which it de-forms and attempts to re-form by
bypassing established fully regulated frameworks for online assembly and
collective activities.

In conclusion, as poetry---and specifically the computational poetry
work done within the \emph{Graph Poem} project---has proved useful
before in cross-disciplinary approaches that provided NLP and DH
outcomes relevant beyond the genre-related tasks and even beyond the
attendant literary concerns, the deployment and further development of
those results in performance helped to cast light on a new dimension of
such research, the social one. While the social relevance and potential
impact of poetry is far from being the invention of our (post)digital
society and our culture(s) of connectivity, new challenges,
opportunities, and accordingly fine-tuned approaches emerged as possible
in digital space and media. As outlined above, they mainly have to do
with collective unconventional (alternative platform) data curation that
will shape community in/as performance---\emph{data-commoning}---and
deploying network applications that disclose and subvert the ubiquitous
control informing the Web by inscriptively and processually disrupting
and deforming established frameworks for online social activity and
assembly or, in one word, by means of \emph{webformance}.

We drew in considering both the social-political and theoretical
implications of such computational performance poetics on recent
reframings of the concept of Umwelt as well as Simondon's philosophy of
technics, particularly as revisited and adapted to poetry by Brian Kim
Stefans. The concept of individuation---as inextricably intertwined with
the one of milieu---turned out remarkably helpful in this endeavor. It
is just that unlike Stefans for instance, our NLP and graph-theory-based
approach to poems as inscriptive performances in digital space and
media, while validating the fundamentally technological nature of poems,
reached different conclusions regarding their interaction with, and
generation of, their milieus. The technical elements making up a poem
emerged therefore not only as not confined in the poem and away from the
environment, but actually the very elements that enact its
(non-individual) individuation and the shaping of its environment in
both poetic and apoetic ways. The latter ensure in digital space and
media the potentially pervasive social and community relevant impact of
poetry in/as computational performance as it radically exposes and
employs multilayered overlappings and interfusions between humans (as
technological and poetic beings), poems (as performative inscriptions in
digital space), and machines (as computationally implemented algorithms
and artificial intelligence).

\textbf{Funding: }

This work was supported by the Social Sciences and Humanities Research
Council of Canada {grant number 430-2014-00349}.
